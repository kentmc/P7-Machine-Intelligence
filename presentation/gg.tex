\section{\scshape Results}

\begin{frame}
\center \huge \scshape Results
\end{frame}

\begin{frame}
  \frametitle{Results}
  \begin{itemize}
  	\item 3 data sets: 6, 23, 35
  	\item Parameters chosen based on the experiments presented earlier
  	\item 5000 sequences
  	\item Static: 10 initial states, step size 10
  	\item Dynamic: 10 initial states, 5 intermediate BW iterations
  \end{itemize}
\end{frame}

\begin{frame}
  \frametitle{Results} 
  \begin{center}
	\begin{table}[h]
		\centering
		\begin{tabular}{|c|r|r|r|r|r|}
		\hline
		Data set    & \multicolumn{1}{c|}{\textbf{GE}} & \multicolumn{1}{c|}{\textbf{BW}} & \multicolumn{1}{c|}{\textbf{SBW}} & \multicolumn{1}{c|}{\textbf{GS}} & \multicolumn{1}{c|}{\textbf{Goal}} \\ \hline
\textbf{6}  & 115.53                           & 122.60                           & 129.64                            & 114.77                           & 66.98                              		\\ \hline
		\textbf{23} & 26.30                            & 26.16                            & 26.08                             & 26.19                            & 18.41                              		\\ \hline
		\textbf{35} & 49.48                            & 43.36                            & 50.79                             & 44.44                            & 33.78                              		\\ \hline
		\end{tabular}
	\caption{The best scores of each algorithm on the three data sets.}
	\label{table:pautomac-best-score}
	\end{table}
  \end{center}
\end{frame}
	
\begin{frame}
  \frametitle{Results} 
  \begin{center}
	\begin{figure}
\centering
\begin{tikzpicture}[scale=0.80]
	\pgfplotsset{every axis legend/.append style={ 
		at={(0.5,1.1)},
		anchor=south}}
	\begin{axis}[
			xmin = 0,
			xmax = 100,
			ymin = 18,
			ymax = 38,
			cycle list name=color list,
			xlabel = States,
            	ylabel = Perplexity (lower is better),
            	legend columns=-1,
            	legend entries={GE, BW, SBW, GS, Goal},
			legend style={/tikz/every even column/.append style={column sep=0.3cm}},
			cycle list name=exotic]
		
		\addplot+[mark=none]table[x=States, y=GE, col sep=tab]
		{graphdata/pautomac-competition-results-dataset-23.csv};
		
		\addplot+[mark=none]table[x=States, y=BW, col sep=tab]
		{graphdata/pautomac-competition-results-dataset-23.csv};
		
		\addplot+[mark=none]table[x=States, y=SBW, col sep=tab]
		{graphdata/pautomac-competition-results-dataset-23.csv};
		
		\addplot+[mark=none]table[x=States, y=GS, col sep=tab]
		{graphdata/pautomac-competition-results-dataset-23.csv};
		
		\addplot+[mark=none]table[x=States, y=Goal, col sep=tab]
		{graphdata/pautomac-competition-results-dataset-23.csv};
	\end{axis}
\end{tikzpicture}
\caption{The different algorithm's score on \emph{PAutomaC}'s 23rd data set, according to the perplexity measure used in the competition.}
\label{fig:pautomac-competition-23}
\end{figure}
  \end{center}
\end{frame}


\section{\scshape Conclusion}
\subsection{Conclusion}
\begin{frame}
\center \huge \scshape Conclusion
\end{frame}

\begin{frame}
  \frametitle{Conclusion}
  \begin{itemize}
  	\item Analysis of different approaches using a sparse transition matrix in the BW algorithm
  	\item Dynamic algorithms: about as good results as BW when not better
  	\item Sparse matrix ensured by adding states iteratively
  \end{itemize}
\end{frame}

\begin{frame}
  \frametitle{Conclusion: Pros and Cons}
  \begin{center}
	\begin{table}[h]
  \resizebox{\textwidth}{!}{ 
\begin{tabular}{|l|l|l|}
\hline
Algorithm & Advantages                                                                                & Drawbacks                                                                                 \\ \hline
GE        & \begin{tabular}[c]{@{}l@{}}- Rather fast\\ - Good results\end{tabular}                    & \begin{tabular}[c]{@{}l@{}}- Usually needs more states to \\   start to compare\end{tabular} \\ \hline
GS        & \begin{tabular}[c]{@{}l@{}}- Splitting heuristic\\ - Best results, but...\end{tabular}    & \begin{tabular}[c]{@{}l@{}}- Runs on a dense matrix\\ - ...unusual behaviour\end{tabular} \\ \hline
SBW       & \begin{tabular}[c]{@{}l@{}}- Computational speed-up\\ - Room for improvement\end{tabular} & \begin{tabular}[c]{@{}l@{}}- Results slightly worse than BW\\ - No speed-up on some data sets\end{tabular} \\ \hline
\end{tabular}
}
\end{table}
  \end{center}
\end{frame}

\begin{frame}
  \frametitle{Conclusion: Experiments conducted}
  \begin{itemize}
  	\item 3 types of experiment
  	\item 4 algorithms
  \end{itemize}
\end{frame}

\begin{frame}
  \frametitle{Conclusion: First experiments}
  \begin{itemize}
  	\item Extremely varied behaviour on different data sets
  	\item Three available parameters for results analysis
  	\item Hypotheses, but no definite clear pattern
  	\item In general, results on par with BW (sometimes outperforming it)
  \end{itemize}
\end{frame}

\begin{frame}
  \frametitle{Conclusion: Second experiments}
  \begin{itemize}
  	\item Speed comparison between: BW, SBW, GE
  	\item 8-hour run
  	\item GE dominating both in speed and in score
  	\item SBW unstable: random matrix causing non-deterministic behaviour (suggests buiding a data-derived one instead)
  \end{itemize}
\end{frame}

\begin{frame}
  \frametitle{Conclusion: Third experiments}
  \begin{itemize}
  	\item Comparison using PAutomaC scores
  	\item Unsatisfactory results, but...
  	\item ...due to a different paradigm used: no stopping probabilities have been used
  \end{itemize}
\end{frame}


\section{\scshape Future Work}
\begin{frame}
\center \huge \scshape Future Work
\end{frame}

\begin{frame}
  \frametitle{Conclusion: Future work}
  \begin{itemize}
  	\item Eliminating possible noise
  	\item Explain the multitude of behaviours, validating or disproving our hypotheses
  	\item Running on more states
  	\item Comparing with the PAutomaC scores, using the same paradigm
  	\item Optimising the sparse transition matrices used in the algorithms
  	\item Making sure underflow and overfitting are avoided
  	\item Looking more in-depth into the other methods we have put aside in this project
  \end{itemize}
\end{frame}
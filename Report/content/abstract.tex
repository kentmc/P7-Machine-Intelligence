\vspace{2in}
\begin{abstract}

This report covers the work done at the 1st semester of the master's program in Computer Science at Aalborg University. The overall goal of the project was to model reality. The report takes its basis from the now finished Pautomac competition, which was an online competition focusing on learning the parameters of probabilistic automatons. This report covers the use of the hidden Markov model (HMM), the forward-backward, Viterbi and Baum-Welch (BW) algorithm. Different approaches for learning model parameters as well as the structure of the HMM has been proposed. Pautomac delivered 48 data sets for the parameter learning task, which were generated by probabilistic models. A subset of those were used for 12 experiments, in an attempt to study how BW and the proposed algorithms behave across data sets with significantly different parameters and structure. The results from the different data sets were overall not very clear. Other experiments were performed on sparse models, which showed a clear benefit from using sparse models when learning under a time constraint.

\end{abstract} 

\vspace{2in}
\begin{abstract}

This report covers the work done at the 1. candidate semester at Aalborg University. The goal of the project was to learn how to model reality through some technology. The report takes it basis from the Pautomac competition, which was an online parameter learning competition, where the goal of the competition was to use probabilistic automata to learn the approximate distribution over strings. This report covers the use of the hidden Markov model (HMM), the forward-backward, Viterbi and Baum-Welch (BW) algorithm. Different approaches for learning model parameters as well as the structure of the HMM. Pautomac delivered 48 data sets for the parameter learning task, where a subset were used to generate 12 experiment, in an attempt to study how BW and new presented algorithms behave across data set, that were generated by probabilistic model with significantly different parameters and structures. The results from the different data sets were overall not clear. Other experiments were done on sparsity of models, which showed a clear benefit from using sparse models when learning under a time constraint.

\end{abstract} 

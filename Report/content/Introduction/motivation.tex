\section{Motivation}
Sometimes we are faced with a system which underlying behaviour is to us unknown. We can however observe a sequence of outputs generated by the system. One example is within speech recognition \cite{Rabiner89hmm} where a computer can observe a change in pitch over time when a particular word is pronounced. However, the same word can be pronounced in many different ways, so it is useful for the computer to have a model that describes the different pronunciations in order to recognize spoken words. Many similar problems exists, such as recognizing patterns within \gls{dna} strings or proteins in bioinformatics\cite{Sakakibara2005}.

When modelling such hidden systems, the common approach is to use a probabilistic model that defines a probabilistic distribution over all observable sequences. The model has to be trained, with the goal of finding a set of parameters for the model that makes the observed sequences most likely. An underlying model can be defined and trained in many different ways. Two well known models are the \gls{pfa} \cite{pazintroduction} and the \gls{hmm} \cite{Rabiner89hmm}. However, different models have been proposed as being more suited for different tasks. While the \gls{hmm} seems to be the best choice in many applications, the \gls{pcfg} model is assumed to be stronger for modelling bioinformatic systems\cite{Sakakibara2005}.

In May 2012, a competition with the name Pautomac was launched. The competition aimed to find the best model and learning method on a number of data sets generated by various kinds of models. Since the competition is now over and all results have been published, one can now develop new models or learning methods, and compare its performance to the best methods used in the competition.
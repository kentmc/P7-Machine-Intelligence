\section{Probabilistic Automata}
There exist many different automata for learning observations. A general feature of these automata is that the better an automaton can model observations, the harder it will be to train that automaton. The best known automata are the \gls{hmm} and \gls{pfa}, where this report will focus on the use of \gls{hmm}~\cite{pautomacTR}.
Other candidates were considered for this work, including n-gram, Markov chains and some deterministic counterparts to the \gls{pfa}, however all of these are strictly less powerful than the \gls{pfa}. The \gls{hmm} is known to be equivalent with the \gls{pfa}, which makes them equally strong in terms of their modelling power.

Other potential candidates include \gls{pcfg} and \gls{ma}, where the \gls{pcfg} shows strength in bio-informatics and \gls{ma} is known to be stronger than the \gls{pfa}. However as mentioned in section \ref{sec:pautomac} the automata used to produce the data sets are: \gls{mc}, \gls{dpfa}, \gls{hmm} and \gls{pfa}, where modelling strength beyond \gls{pfa} is not necessary. In practice the \gls{pcfg} would manage the data sets just fine, as the data set are all finite sequences. However in reality an observed machine, like a HMM, might emit symbols in one single infinite sequence, where a model based on context free grammar would not be suitable. The reason for this is that the grammatical rules of \gls{pcfg} has to reach a terminal before a string is recognized, which cannot happen on an infinite string.
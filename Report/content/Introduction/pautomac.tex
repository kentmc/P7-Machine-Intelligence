\section{PautomaC Competition}
\label{sec:pautomac}

PautomaC (Probabilistic Automata learning Competition) was an online competition about learning distributions over strings. For the competition a number of probabilistic automata have each generated a large set of sequences, which have been compiled into a training set and a test set, where the test data were used to evaluate a score for the contestants.
The contestants did not know anything about the automata, which were randomly generated from four unknown parameters.

To reach a score from a learned model, the 1000 sequences of the test set were each given a probability. These probabilities were then compared to the probabilities of the original data generator, using the perplexity measure:

\begin{equation} \label{eq:perplexity}
2^{-(\sum_{x\epsilon TestSet}P_{r_{T}}(x)\times\log(P_{r_{C}}(x))}
\end{equation}

Where $P_{r_{T}}(x)$ represents the real probability of the
sequence $x$, and $P_{r_{C}}$ is a contestant's submitted probability for the sequence $x$.
The contestants could submit an unlimited number of solutions. However only a rank compared to the other contestants, and not a precise score, was given while the competition was running.

\subsubsection{About PautomaC's Automata}
While the contestant did not know anything about the automata, we have the luxury of some more exact parameters for the models, including:

\begin{itemize}
\item Markov Chains
\item Deterministic Probabilistic Finite Automata (DPFA)
\item Probabilistic Finite Automata (PFA)
\item Hidden Markov Models (HMM)
\end{itemize}
All model parameters have been randomly generated based on some values chosen for:
\begin{description}
\item N for the number of states
\item A for the size of the alphabet (observable symbols)
\item $S_s$ for the sparsity of symbols
\item $T_S$ for the sparsity of transitions
\end{description}
Where $N \cdot T_s$ is used as the number of initial states while $N \cdot S_s$ is used as the number of final states. For this report $S$ will denote the size of the alphabet for a given model and $T$ will denote the amount of transitions in percentage, given the state space and maximum possible amount of transitions.
\section{Algorithms}
In this section a number of algorithms are described, which all try to learn the parameters of a \gls{hmm} given a set of training sequences $D$, which contains a number of sequences over $s$ distinct symbols.
In other words, each algorithm will try to find a model that makes the generation of the particular training sequences most likely.
Since a \gls{hmm} can be represented by the use of either matrices or a graph like structure, any of the algorithms may output either a matrix or graph representation of a \gls{hmm}, denoted by $M$ or $G$ respectively.
By $LL(M)$ or $LL(G)$ we denote the likelihood of the training sequences given the model.

The algorithms proposed in this section have been divided into two types.
The static algorithms start with a \gls{hmm} with a particular number of states which will never change.
The dynamic algorithms start with a \gls{hmm} of just a single state, and will dynamically extend the number of states through a number of iterations. 
For the static algorithms, $n$ denotes the number of states to be used.
The dynamic algorithms will by $n$ denote a number of iterations $1, ..., n$, where each iteration extends the model by a single node.
In \ref{fig:alg-hierarchy} the names and classes of our proposed algorithms can be seen. Note that the Baum Welch algorithm has also been included, simply to emphasize which class of algorithms it belongs to. 

\begin{figure}[!h]
\Tree[.Algorithms
		[.{Static size} 
			{Baum Welch}
            {Sparse Baum Welch}
        ]
       	[.{Dynamic size} 
       		{Gamma splitter}
       		{Greedy Extend}
      	]
     ]
\caption{The algorithms used in our experiments}
\label{fig:alg-hierarchy}
\end{figure}

Since the Baum Welch algorithm is guaranteed to never worsen $LL(G)$ when run on $G$, it is used internally by some of the algorithms.
By $BW_t(M, D)$ we denote the \gls{hmm} obtained after running Baum Welch on the \gls{hmm} $M$ using the training sequences $D$, and iterating as long as each iteration increases the likelihood by at least $t$.
By $BW^i(M, D)$ we denote the \gls{hmm} obtained after running $i$ iterations of Baum Welch on the \gls{hmm} $M$ using the training sequences $D$.
If a \gls{hmm} is represented by a graph, we denote it $G$, and the similar notations $BW_t(G, D)$ and $BW^i(G, D)$ are used.

\subsection{Sparse Baum Welch}
This algorithm creates a \gls{hmm} $M$ with $n$ states and $s$ symbols.
All parameters are initialized randomly but with the constraint that each state has exactly $\lceil log(n) \rceil$ outgoing transitions.
Which transitions to discard are chosen randomly before training with \gls{baum-welch} either by $BW_t(M, D)$ or $BW^i(M, D)$.
\subsection{Greedy Extend}
Initially, a graph representation $G$ of a single state \gls{hmm} is created. The single node has initial probability 1, loops to itself with probability 1, and its emission probabilities for each of the $s$ symbols are chosen randomly and normalised.

The following pseudo code describes how the algorithm continuously tries to extend the graph, as long as it improves the likelihood of the data:

\begin{itemize}
\item (1) Repeat $\alpha$ times:
	\begin{itemize}
	\item $G'$ = $(V(G) \cup \{y'\}, E(G))$, where $y'$ is a new node with a random initial probability in range $[0, 1]$ having random emission probabilities for all $s$ symbols, which sums to $1$.
	\item Randomly choose a set of nodes $Y = \{y_1, y_2, ... , y_l\}$ from $V(G')$, where $l = \lceil \log |V(G')| \rceil$ and $\forall a,b: y_a \neq y_b$.
	\item For each $y \in Y$, the transitions $(y, y')$ and $(y', y)$ are added to $E(G')$ with random transition probabilities.
	\item Normalize $G'$.
	\item If $LL(BW^{\beta}(G', D)) > LL(G)$, let $G = LL(BW^{\beta}(G', D))$, go to (1).
	\begin{itemize}
		\item Return $BW_t(G, D)$.
	\end{itemize}
	\end{itemize}
\end{itemize}

\paragraph{Determining the $\alpha$ Value}

The value $\alpha$ determines how many times the algorithm will try to expand the model. If the $\alpha$ attempts cannot improve the model further, it stops and returns the model. By observing the algorithm it is clear that most small model improve with the very first expansion attempt, while larger models need more attempt to find a new state that improves the model. It is difficult to determine exactly how many expansion attempt that can be deemed sufficient, as the amount of random expansion possibilities is very large, however as we shall see later in Chapter \ref{chap:experiment} where 100 runs were used, only few experiments ended because 100 expand attempts without improving.

\paragraph{Determining the $\beta$ Value}

Another big question about the Greedy Extend algorithm, is how the choice of $\beta$ affects the performance of the algorithm.
As $\beta$ denotes the number of Baum Welch iterations to run each time the algorithm attempts to extend the graph, increasing $\beta$ will also increase the run time of the algorithm. It may be the case that better results are achieved when $\beta$ is increased, since more iterations of Baum Welch also means a greater increase in likelihood. However, it could be the case that using many iterations early increases the chance of getting trapped in a local optimum.
An experiment has been conducted of using different values for $\beta$ on data set $1$ from the Pautomac competition. The results can be seen in figure \ref{fig:ge-different-thresholds-tested}. Each line represents the mean value of 5 runs of the Greedy Extend algorithm with the specified number of iterations. Some of the plots have been cut off at a point where one of the runs did not manage to extend beyond a certain number of states.

\begin{figure}[!h]
\begin{centering}
\begin{tikzpicture}
	\pgfplotsset{every axis legend/.append style={ 
		at={(0.5,1.06)},
		anchor=south}}
	\begin{axis}[
			scale = 1.5,
			xlabel = Number of states,
            	ylabel = Score (lower is better),
            	legend columns=-1,
            	legend entries={IT-0, IT-1, IT-2, IT-3, IT-5, IT-10, IT-50},
			legend style={/tikz/every even column/.append style={column sep=0.3cm}}]
		
		\addplot+[mark=none]table[x=States, y=IT-0, col sep=tab]
		{content/Models/Algorithms/ge-intermediate-iterations-test.csv};
		
		\addplot+[mark=none]table[x=States, y=IT-1, col sep=tab]
		{content/Models/Algorithms/ge-intermediate-iterations-test.csv};
		
		\addplot+[mark=none]table[x=States, y=IT-2, col sep=tab]
		{content/Models/Algorithms/ge-intermediate-iterations-test.csv};
		
		\addplot+[mark=none]table[x=States, y=IT-3, col sep=tab]
		{content/Models/Algorithms/ge-intermediate-iterations-test.csv};

		\addplot+[mark=none]table[x=States, y=IT-5, col sep=tab]
		{content/Models/Algorithms/ge-intermediate-iterations-test.csv};
		
		\addplot+[mark=none]table[x=States, y=IT-10, col sep=tab]
		{content/Models/Algorithms/ge-intermediate-iterations-test.csv};
		
		\addplot+[mark=none]table[x=States, y=IT-50, col sep=tab]
		{content/Models/Algorithms/ge-intermediate-iterations-test.csv};
	\end{axis}
\end{tikzpicture} 
\caption{Test of different values for $\beta$ while running the Greedy Extend algorithm.}
\label{fig:ge-different-thresholds-tested} 
\end{centering}
\end{figure}

The figure shows surprisingly that using no iterations is somehow better than using just a single iteration. However, using 5 or more iterations is better than not using any iterations at all.
When using 5 iterations, only a single run did not reach 50 states (it stopped at 49).
We later choose to conduct further experiments with $\beta = 10$, since all runs with 10 iterations reached 50 states, and it looks like the performance does not change significantly when increasing the number of iterations beyond 10.
\subsection{State Splitting Approach}

Another dynamic size \gls{hmm} learning approach was to construct a greedy heuristics to ``split'' states. Several versions and concepts were attempted whilst maintaining the \gls{baum-welch} as a basis for the approach, running it repeatedly on the growing model to ensure convergence. The initial experiments were splitting states until a given number of states ($n$) was reached, but a threshold ($\theta$) mechanic was employed at a later time in attempt to reduce the dependency on the prior knowledge of number of states.

The greedy state splitting algorithm consists of two main modules, with possible extensions. The main modules include the heuristics to identify the state (or multiple states) to split and the splitting algorithm itself. During the experiments, two additional concepts, not fundamental for the state splitting itself, were also explored - edge cutting and state removal.

\subsubsection{State Split Mechanics}
Two main approaches to splitting the states were considered for the state splitting algorithm. One being very simple, just producing a copy of the state to split (\emph{Clone split}) and a much more elaborate approach considering the topology of the state inside the hidden state graph (\emph{Distribution split}).

\subsubsection{State Identification Heuristics}
Similarly to state splitting itself, two main approaches were explored for identifying the best state to split. The first one utilised the \gls{viterbi} thus named the \emph{Viterbi heuristic}, whilst the second one uses the $\gamma_t(i)$ variables computed during the \gls{baum-welch} algorithm and was therefore named the \emph{Gamma heuristic}.

Both of the heuristics compute a score $\zeta:S \rightarrow \mathcal{R}$ for each hidden state of the model, that can later be utilised to determine the state to split.

\paragraph{Viterbi Heuristic}
The \emph{Viterbi heuristic} computes the score $\zeta$ for each hidden state as a probability of producing the correct symbol in the validation sequences for which it belongs to the corresponding most probable hidden state sequence as determined by the \gls{viterbi}.

In more formal terms, the computation of the score $\zeta(s)$ for each $s \in S$ and a given \gls{hmm} $\lambda = (\mathbf{A}, \mathbf{B}, \boldsymbol{\pi})$ can be described by the following procedure:
\begin{enumerate}
	\item For each signal $\mathbf{O}\in V$ a corresponding most probable hidden state sequence is computed using the \gls{viterbi}: $\mathbf{Q}=\mathcal{V}_G(\mathbf{O})$.
	\item For each state $s\in S$ determine the significant positions in the hidden state sequences given by \gls{viterbi}:
	$$\forall s\in S,\forall \mathbf{O}=(o_1, ..., o_T)\in V: \tau_{s, \lambda}^\mathbf{O}=\{t\in\{1, ..., T\}|\mathbf{Q}_t=s\}$$
	\item For each state $s \in S$  and $\mathbf{O}\in V$ compute the partial score (performance) $\hat\zeta_{\mathbf{O}}(s)$ as the average probability of producing the expected observable symbol in accordance to the signal $\mathbf{O}$ over all the significant positions for the given state $s$ and signal $\mathbf{O}$:
	$$\forall s\in S,\forall \mathbf{O}\in V: \hat{\zeta}_{\mathbf{O}}(s) = \frac{\sum_{t\in\tau_{s, \lambda}^{\mathbf{O}}}b_s(o_t)}{|\tau_{s, \lambda}^{\mathbf{O}}|}$$
	\item Finally, compute the $\zeta$ score for each state $s\in S$ as a sum of all the partial scores for the given state weighted by the probabilities of generating the associated signals by the corresponding hidden state sequences:
	$$\forall s\in S: \zeta(s)=\sum_{\mathbf{O}\in V}P(\mathbf{Q}|\mathbf{O}, \lambda)\hat\zeta_{s, \lambda}^{\mathbf{O}}$$
\end{enumerate}

The obtained $\zeta$ scores the states of the \gls{hmm} based on their ``performance'' for the tasks they are most likely to perform. As such the state with the lowest score is determined to be the worst performing node $w$: $$w = \argmin_{s\in S}(\zeta(s))$$

This node is deemed to be the worst performing one as a result of being involved in the generation of many of the signals in the validation set $V$. As such, it seems meaningful to split the node into two, in order to share the extensive workload and increase performance.

The \emph{Viterbi heuristic} can be straightforwardly extended to identify more than just one state to split, thereby producing a set of the worst performing nodes $\mathcal{W}$. The set can be constructed iteratively starting with $\mathcal{W} = \emptyset$ as:
$$\mathcal{W} = \mathcal{W} \cup \{\argmin_{s\in S\setminus \mathcal{W}}(\zeta(s))\}$$
until the $|\mathcal{W}|$ equals the desired number of states to split.

A further modification of the \emph{Viterbi heuristic} was considered to incorporate the use of a splitting threshold $\theta$ instead of a maximum number of states. For this purpose a normalised version of the score $\overline{\zeta}$ was introduced:
$$\overline{\zeta}(s) = \frac{\zeta(s)}{\sum_{s\in S}\zeta(s)}$$

The states to split $\mathcal{W}$ were thus determined as all states that scored below the given threshold $\theta$:
$$\mathcal{W} = \{s\in S|\overline\zeta(s) < \theta\}$$

More improvements to the \emph{Viterbi heuristic} were considered, mainly including the \emph{n-step Viterbi heuristic} that would have computed the score on not only the output probability in the given significant position, but also on the probability of correctly outputting the next $n -1$ symbols of the given signal - starting from the explored state - to further increase precision. The above described version of the \emph{Viterbi heuristic} would be considered \emph{1-step Viterbi heuristic} in this context. This approach however remains untested due to preference of the \emph{Gamma heuristic} and can be considered for future work.

\paragraph{Gamma Heurisitic}

\todo{Finish the state splitting section.}

%This algorithm utilises a greedy concept to grow the hidden state space of a \gls{hmm} whilst maintaining a sparse transition matrix to preserve computability in large state spaces. The Greedy %State Splitting relies heavily on the existing \gls{baum-welch} calling it repeatedly to achieve convergence.

%Let $(n, s, \epsilon, D, V)$ be the input vector of the Greedy State Splitting algorithm where: $n, s \in \mathbb{N}, \epsilon\in[0,1], D$ is the training data set and $V$ is the validation data %set. The greedy State Splitting algorithm starts with two vertex complete graph $G=K_2$ for the hidden state space with all the parameters randomised. Afterwards it iterates through the %following phases until $|V(G)| = n$:

%\begin{itemize}
%	\item[] Phase 1, state splitting
%	\item[1)] For each observable state sequence $\mathbf{O} \in V$ a corresponding most probable hidden state sequence is computed using \gls{viterbi}: %$\mathbf{Q}=\mathcal{V}_G(\mathbf{O})$.
%	\item[2)] For each vertex $v \in V(G)$ compute a score $s(v)$ as the probability the given vertex outputs the desired output symbol according to the precomputed hidden state %sequences weighted by the probability of these sequences. In more formal terms, let $\theta_G(\mathbf{O}=(o_1,...,o_T), v) = \{t\in\{1, ..., T\}|\mathbf{Q}_t=v\}$ then: $$s(v) = %\sum_{\mathbf{O}\in V}P(\mathbf{Q}|\mathbf{O},G) \frac{\sum_{t \in \theta_G(\mathbf{O}, v)}b_v(o_t)}{|\theta_G(\mathbf{O}, v)|}$$
%	\item[3)] Find the ``weakest'' vertex: $$w = \argmin_{v\in V(G)}\{s(v)\}$$.
%	\item[4)] Create a new graph $G' = G\cup \{w'\}$ where $w'$ is a new vertex such that: $\forall v\in V(G): a_{w'v} = a_{wv} \land a_{vw'} = a_{vw}$, $\forall \sigma \in \Sigma: %b_{w'}(\sigma) = b_w(\sigma)$ and $\pi(w') = \pi(w)$.
%	\item[5)] Normalise $G'$ so all the probabilities sum up to $1$.
%	\item[] Phase 2, edge cutting
%	\item[6)] For each edge $e = (v_1,v_2)\in E(G')$ check if the edge probability is lower then the given threshold: $a_{v-1,v_2}<\epsilon$. If so, remove the edge from the graph %($a_{v-1,v_2} = 0$).
%	\item[] Phase 3, re-estimation of the model parameters.
%	\item[7)] Run \gls{baum-welch} to re-learn the model parameters of the new model: $G = BW_t(G', D)$.
%\end{itemize}

%The algorithm has also been considered in a ``strict'' variation at first, where the edge cutting phase did not depend on the parameter $\epsilon$ but instead a constant out-degree was %maintained for all vertices, namely the size of the output symbol alphabet $s = |\Sigma|$. The early results however showed, that the strict out-degree variation is outperformed by the %$\epsilon$ threshold.
\subsubsection{Gamma State Splitting}

% finding the most used state for a given sentence.
$$q' = \argmax_{q \in Q}\sum_{o \in O}\sum_{t=0}^{T}\gamma_t(q)$$ 

% find all possible transitions from q'
$$z = \{a_{q'J} \in A \vert a_{q'J} \neq 0\}$$

% compute average transition probability
$$z_{arg} = \frac{\sum_{a_{q'J} \in z} a_{q'J}}{\vert z \vert}$$

% compute the absolute error sum of all transitions. 
$$AbsE = \sum_{a_{q'J} \in z} \vert a_{q'J} - z_{arg} \vert$$

% if the absolute error is below the threshold value epsilon the
% distribution is fairly uniform and the state q' is split into q_1 and q_2.
$$Q' = \begin{cases}
	Q-\{q'\} \cup \{ q_1,q_2 \} &\text{if } AbsE > \epsilon_1 \\
	Q	&\text{otherwise}
\end{cases}$$

% 
$$A_1 = \{a_{q_1i} \vert a_{q_1i} = a_{q'i} \wedge a_{q'i} \in A \wedge a_{q'i} >0 \wedge i \in Q' - \{q_1\}\}$$

$$A_2 = \{a_{q_2i} \vert a_{q_2i} = a_{q'i} \wedge a_{q'i} \in A \wedge a_{q'i} > 0 \wedge i \in Q' - \{q_2\}\}$$

$$A_1 \cap A_2 = \emptyset$$

$$\vert A_1 \vert = \vert A_2 \vert$$

$$A' = A_1 \cup A_2$$

$$\sum_{a_{q_1i} \in A_1} a_{q_1i} = \sum_{a_{q_2J} \in A_1} a_{q_1J}$$

$$A_1 = \{a_{q_1i} \vert a_{q_1i} \in A \wedge a_{q_1i} \notin A_2\}$$
	$$A_1 = \{a_{q_2i} \vert a_{q_2i} \in A \wedge a_{q_2i} \notin A_1\}$$





%%%%%%%%%%%%%%%%%%%%%%%%%%%%%%%%%%%%%%%%%
% Arsclassica Article
% LaTeX Template
% Version 1.1 (10/6/14)
%
% This template has been downloaded from:
% http://www.LaTeXTemplates.com
%
% Original author:
% Lorenzo Pantieri (http://www.lorenzopantieri.net) with extensive modifications by:
% Vel (vel@latextemplates.com)
%
% License:
% CC BY-NC-SA 3.0 (http://creativecommons.org/licenses/by-nc-sa/3.0/)
%
%%%%%%%%%%%%%%%%%%%%%%%%%%%%%%%%%%%%%%%%%

%----------------------------------------------------------------------------------------
%	PACKAGES AND OTHER DOCUMENT CONFIGURATIONS
%----------------------------------------------------------------------------------------

\documentclass[
10pt, % Main document font size
a4paper, % Paper type, use 'letterpaper' for US Letter paper
oneside, % One page layout (no page indentation)
%twoside, % Two page layout (page indentation for binding and different headers)
headinclude,footinclude, % Extra spacing for the header and footer
BCOR5mm, % Binding correction
]{scrartcl}

\input{structure.tex} % Include the structure.tex file which specified the document structure and layout

\hyphenation{Fortran hy-phen-ation} % Specify custom hyphenation points in words with dashes where you would like hyphenation to occur, or alternatively, don't put any dashes in a word to stop hyphenation altogether

%----------------------------------------------------------------------------------------
%	TITLE AND AUTHOR(S)
%----------------------------------------------------------------------------------------

\title{\normalfont\spacedallcaps{Article Title}} % The article title

\author{\spacedlowsmallcaps{John Smith* \& James Smith\textsuperscript{1}}} % The article author(s) - author affiliations need to be specified in the AUTHOR AFFILIATIONS block

\date{} % An optional date to appear under the author(s)

%----------------------------------------------------------------------------------------

%\subsubsection{Variable length n-gram Extend}
\begin{document}
This algorithm consists of two main modules. The first one creates a rough initial graph $G$, the second one extends it. The extend algorithm used in the experiments was the Greedy Extend one {ref to greedy}%\ref{chap:experiments}.

When training on a new sequence, several state sequences may produce the same symbol sequence. The intuition is to create a rough initial graph $G$ which disregards this possibility. Once created, the extending algorithm gradually smoothes this rough approximation.

In the initial graph $G$, all the states have an initial probability of 0, except for one. In this single state have been merged all the possible initial states. The extend algorithm can then increase the number of possible initial states, normalising the initial probabilities before running the Baum Welch algorithm.

The graph $G$is created following a variable length $n$-gram approach. A typical $n$-gram approach is to learn then detect subsequences of length $n$. Just like $n$-grams, this variable length n-gram splits the sequence into subsequences, then tries to match a set of states to each.

However, instead of having a fixed length for those subsequences, our variable length n-gram fetches the largest set of states able to match the sequence from the start. It then splits the whole sequence into two subsequences, and repeats the process on the second subsequence. Once the sequence has been completely split into subsequences, each last state of a subsequence is made to be able to transition to the first state of the next subsequence. In pseudo-code:

\begin{itemize}
	\item $s$ = $s_1 + s'$, where $s$ is the whole sequence, $s_1$ the largest set of states able to match $s$, and $s'$ the rest of the sequence.
	\item Repeat until the end of the sequence:
	\begin{itemize}
		\item $s'$ = $s_i$ + $s''$
		\item increment i and set $s'$ to now be $s''$
	\end{itemize}
	\item For each $s_i \in S = \{s_1, s_2, ... , s_n-1\}$, a transition is added between the last state of $s_i$ and the first state of $s_i+1$, with random transition probabilities.
	\item Normalise $G$.
\end{itemize}

Experiments have shown this can sometimes lead to several states having very few transitions. Keeping the same intuition in mind, those states are all merged into one - if their number of transitions is less than $alpha$.


\end{document}
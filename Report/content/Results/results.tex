\FloatBarrier

\chapter{Results}\label{sec:results}

\paragraph{State Space Experiments}

The purpose of the state experiments are to show how \gls{bw} and our own algorithms and models perform in relation to the state space.\\

	\begin{figure}[h]
\begin{tabular*}{\textwidth}{@{}cc@{}}
\begin{minipage}{\dimexpr0.55\textwidth-2\tabcolsep}
\centering
\textbf{Data set: 6, States: 19}\par\medskip
\begin{tikzpicture}
	\pgfplotsset{every axis legend/.append style={ 
		at={(0.5,1.1)},
		anchor=south}}
	\begin{axis}[
			scaled ticks = true,
			scaled y ticks=base 10:-4,
			xlabel = Number of states,
            	ylabel = Log likelihood,
            	legend columns=-1,
            	legend entries={BW, SBW, GE, GS},
			legend style={/tikz/every even column/.append style={column sep=0.3cm}}]
			
		\addplot+table[x=States, y=BW, col sep=tab]
		{content/Experiments/graphdata/set6.csv};
%		\addlegendentry{\textbf{BW}}
		
		\addplot+table[x=States, y=SBW, col sep=tab]
		{content/Experiments/graphdata/set6.csv};
%		\addlegendentry{\textbf{Sparse BW}}
		
		\addplot+table[x=States, y=GE, col sep=tab]
		{content/Experiments/graphdata/set6.csv};
		
		\addplot+table[x=States, y=Padawan, col sep=tab]
		{content/Experiments/graphdata/set6.csv};
%		\addlegendentry{\textbf{Greedy Extend}}
	\end{axis}
\end{tikzpicture} 
\end{minipage}% 
&
\begin{minipage}{\dimexpr0.55\textwidth-2\tabcolsep}
\centering
\textbf{Data set: 23, States: 33}\par\medskip
\begin{tikzpicture}
	\pgfplotsset{every axis legend/.append style={ 
		at={(0.5,1.1)},
		anchor=south}}
	\begin{axis}[
			scaled ticks = true,
			scaled y ticks=base 10:-4,
			xlabel = Number of states,
            	ylabel = Log likelihood,
            	legend columns=-1,
            	legend entries={BW, SBW, GE, GS},
			legend style={/tikz/every even column/.append style={column sep=0.3cm}}]
			
		\addplot+table[x=States, y=BW, col sep=tab]
		{content/Experiments/graphdata/set23.csv};
%		\addlegendentry{\textbf{BW}}
		
		\addplot+table[x=States, y=SBW, col sep=tab]
		{content/Experiments/graphdata/set23.csv};
%		\addlegendentry{\textbf{Sparse BW}}
		
		\addplot+table[x=States, y=GE, col sep=tab]
		{content/Experiments/graphdata/set23.csv};
		
		\addplot+table[x=States, y=Padawan, col sep=tab]
		{content/Experiments/graphdata/set23.csv};
%		\addlegendentry{\textbf{Greedy Extend}}
	\end{axis}
\end{tikzpicture} 
\end{minipage}
\\
\begin{minipage}[t]{\dimexpr0.6\textwidth-2\tabcolsep}
\end{minipage}
&
\begin{minipage}[t]{\dimexpr0.4\textwidth-2\tabcolsep}
\end{minipage}
\end{tabular*}%
\\
\begin{tabular*}{\textwidth}{@{}cc@{}}
\begin{minipage}{\dimexpr0.55\textwidth-2\tabcolsep}
\centering
\textbf{Data set: 41, States: 54}\par\medskip
\begin{tikzpicture}
	\pgfplotsset{every axis legend/.append style={ 
		at={(0.5,1.1)},
		anchor=south}}
	\begin{axis}[
			scaled ticks = true,
			scaled y ticks=base 10:-4,
			xlabel = Number of states,
            	ylabel = Log likelihood,
            	legend columns=-1,
            	legend entries={BW, SBW, GE, GS},
			legend style={/tikz/every even column/.append style={column sep=0.3cm}}]
			
		\addplot+table[x=States, y=BW, col sep=tab]
		{content/Experiments/graphdata/set41.csv};
%		\addlegendentry{\textbf{BW}}
		
		\addplot+table[x=States, y=SBW, col sep=tab]
		{content/Experiments/graphdata/set41.csv};
%		\addlegendentry{\textbf{Sparse BW}}
		
		\addplot+table[x=States, y=GE, col sep=tab]
		{content/Experiments/graphdata/set41.csv};
		
		\addplot+table[x=States, y=Padawan, col sep=tab]
		{content/Experiments/graphdata/set41.csv};
%		\addlegendentry{\textbf{Greedy Extend}}
	\end{axis}
\end{tikzpicture} 
\end{minipage}% 
&
\begin{minipage}{\dimexpr0.55\textwidth-2\tabcolsep}
\centering
\textbf{Data set: 1, States: 64}\par\medskip
\begin{tikzpicture}
	\pgfplotsset{every axis legend/.append style={ 
		at={(0.5,1.1)},
		anchor=south}}
	\begin{axis}[
			scaled ticks = true,
			scaled y ticks=base 10:-4,
			xlabel = Number of states,
            	ylabel = Log likelihood,
            	legend columns=-1,
            	legend entries={BW, SBW, GE, GS},
			legend style={/tikz/every even column/.append style={column sep=0.3cm}}]
			
		\addplot+table[x=States, y=BW, col sep=tab]
		{content/Experiments/graphdata/set1.csv};
%		\addlegendentry{\textbf{BW}}
		
		\addplot+table[x=States, y=SBW, col sep=tab]
		{content/Experiments/graphdata/set1.csv};
%		\addlegendentry{\textbf{Sparse BW}}
		
		\addplot+table[x=States, y=GE, col sep=tab]
		{content/Experiments/graphdata/set1.csv};
		
		\addplot+table[x=States, y=Padawan, col sep=tab]
		{content/Experiments/graphdata/set1.csv};
%		\addlegendentry{\textbf{Greedy Extend}}
	\end{axis}
\end{tikzpicture} 
\end{minipage}
\\
\begin{minipage}[t]{\dimexpr0.6\textwidth-2\tabcolsep}
\end{minipage}
\end{tabular*}%
\label{fig:states}\caption{Four data sets produced by models with 19 to 64 states}
\end{figure}

Initial observations
\begin{itemize}
\item GE performs the best, except dataset 1. It seems like GE would surpass BW with a larger state space.
\item SBW performs worse than BW, note that dataset 41 has a fairly small y-axis, where the best GE result is about 800 points. We estimate an error margin of 100-200 points. Without more runs for each experiment, it is impossible to determine an exact error margin.
\item In general we seem to 
\end{itemize}


\paragraph{Transition Density Experiments}

The purpose of the transition density experiments are to show how \gls{bw} and our own algorithms and models perform in relation to the transition density.\\

	\begin{tabular*}{\textwidth}{@{}cc@{}}
\begin{minipage}{\dimexpr0.55\textwidth-2\tabcolsep}
\centering
\textbf{Dataset: 36, Density: 7.4\%}\par\medskip
\begin{tikzpicture}
	\pgfplotsset{every axis legend/.append style={ 
		at={(0.55,1.03)},
		anchor=south}}
	\begin{axis}[
			scaled ticks = true,
			scaled y ticks=base 10:-4,
			xlabel = Number of states,
            	ylabel = Score (lower is better),
            	legend columns=-1,
            	legend entries={BW, SBW, GE},
			legend style={/tikz/every even column/.append style={column sep=0.3cm}}]
		
		\addplot+table[x=States, y=BW, col sep=tab]
		{content/Experiments/graphdata/set36.csv};
%		\addlegendentry{\textbf{BW}}
		
		\addplot+table[x=States, y=SBW, col sep=tab]
		{content/Experiments/graphdata/set36.csv};
%		\addlegendentry{\textbf{Sparse BW}}
		
		\addplot+table[x=States, y=GE, col sep=tab]
		{content/Experiments/graphdata/set36.csv};
%		\addlegendentry{\textbf{Greedy Extend}}
	\end{axis}
\end{tikzpicture} 
\label{fig:dataset36}
\end{minipage}% 
&
\begin{minipage}{\dimexpr0.55\textwidth-2\tabcolsep}
\centering
\textbf{Dataset: 8, Density: 16.8\%}\par\medskip
\begin{tikzpicture}
	\pgfplotsset{every axis legend/.append style={ 
		at={(0.55,1.03)},
		anchor=south}}
	\begin{axis}[
			scaled ticks = true,
			scaled y ticks=base 10:-4,
			xlabel = Number of states,
            	ylabel = Score (lower is better),
            	legend columns=-1,
            	legend entries={BW, SBW, GE},
			legend style={/tikz/every even column/.append style={column sep=0.3cm}}]
		
		\addplot+table[x=States, y=BW, col sep=tab]
		{content/Experiments/graphdata/set8.csv};
%		\addlegendentry{\textbf{BW}}
		
		\addplot+table[x=States, y=SBW, col sep=tab]
		{content/Experiments/graphdata/set8.csv};
%		\addlegendentry{\textbf{Sparse BW}}
		
		\addplot+table[x=States, y=GE, col sep=tab]
		{content/Experiments/graphdata/set8.csv};
%		\addlegendentry{\textbf{Greedy Extend}}
	\end{axis}
\end{tikzpicture} 
\label{fig:dataset8}
\end{minipage}
\\
\begin{minipage}[t]{\dimexpr0.6\textwidth-2\tabcolsep}
\end{minipage}
&
\begin{minipage}[t]{\dimexpr0.4\textwidth-2\tabcolsep}
\end{minipage}
\end{tabular*}%
\\
\begin{tabular*}{\textwidth}{@{}cc@{}}
\begin{minipage}{\dimexpr0.55\textwidth-2\tabcolsep}
\centering
\textbf{Dataset: 43, Density: 40.2\%}\par\medskip
\begin{tikzpicture}
	\pgfplotsset{every axis legend/.append style={ 
		at={(0.55,1.03)},
		anchor=south}}
	\begin{axis}[
			scaled ticks = true,
			scaled y ticks=base 10:-4,
			xlabel = Number of states,
            	ylabel = Score (lower is better),
            	legend columns=-1,
            	legend entries={BW, SBW, GE},
			legend style={/tikz/every even column/.append style={column sep=0.3cm}}]
			
		\addplot+table[x=States, y=BW, col sep=tab]
		{content/Experiments/graphdata/set43.csv};
%		\addlegendentry{\textbf{BW}}
		
		\addplot+table[x=States, y=SBW, col sep=tab]
		{content/Experiments/graphdata/set43.csv};
%		\addlegendentry{\textbf{Sparse BW}}
		
		\addplot+table[x=States, y=GE, col sep=tab]
		{content/Experiments/graphdata/set43.csv};
%		\addlegendentry{\textbf{Greedy Extend}}
	\end{axis}
\end{tikzpicture} 
\label{fig:dataset43}
\end{minipage}% 
&
\begin{minipage}{\dimexpr0.55\textwidth-2\tabcolsep}
\centering
\textbf{Dataset: 37, Density: 50\%}\par\medskip
\begin{tikzpicture}
	\pgfplotsset{every axis legend/.append style={ 
		at={(0.55,1.03)},
		anchor=south}}
	\begin{axis}[
			scaled ticks = true,
			scaled y ticks=base 10:-4,
			xlabel = Number of states,
            	ylabel = Score (lower is better),
            	legend columns=-1,
            	legend entries={BW, SBW, GE},
			legend style={/tikz/every even column/.append style={column sep=0.3cm}}]
			
		\addplot+table[x=States, y=BW, col sep=tab]
		{content/Experiments/graphdata/set37.csv};
%		\addlegendentry{\textbf{BW}}
		
		\addplot+table[x=States, y=SBW, col sep=tab]
		{content/Experiments/graphdata/set37.csv};
%		\addlegendentry{\textbf{Sparse BW}}
		
		\addplot+table[x=States, y=GE, col sep=tab]
		{content/Experiments/graphdata/set37.csv};
%		\addlegendentry{\textbf{Greedy Extend}}
	\end{axis}
\end{tikzpicture} 
\label{fig:dataset37}
\end{minipage}
\\
\begin{minipage}[t]{\dimexpr0.6\textwidth-2\tabcolsep}
\end{minipage}
&
\begin{minipage}[t]{\dimexpr0.4\textwidth-2\tabcolsep}
\end{minipage}
\end{tabular*}%

Initial observations
\begin{itemize}
\item GE performs the best
\item 
\item 
\end{itemize}

\paragraph{Symbol Alphabet Size Experiments}
The purpose of the s are to show how \gls{bw} and our own algorithms and models perform in relation to the size of the emission symbol alphabet size.\\

	\begin{tabular*}{\textwidth}{@{}cc@{}}
\begin{minipage}{\dimexpr0.55\textwidth-2\tabcolsep}
\centering
\textbf{Dataset: 32, Symbols: 4}\par\medskip
\begin{tikzpicture}
	\pgfplotsset{every axis legend/.append style={ 
		at={(0.55,1.03)},
		anchor=south}}
	\begin{axis}[
			scaled ticks = true,
			scaled y ticks=base 10:-4,
			xlabel = Number of states,
            	ylabel = Log likelihood,
            	legend columns=-1,
            	legend entries={BW, SBW, GE},
			legend style={/tikz/every even column/.append style={column sep=0.3cm}}]
			
		\addplot+table[x=States, y=BW, col sep=tab]
		{content/Experiments/graphdata/set32.csv};
%		\addlegendentry{\textbf{BW}}
		
		\addplot+table[x=States, y=SBW, col sep=tab]
		{content/Experiments/graphdata/set32.csv};
%		\addlegendentry{\textbf{Sparse BW}}
		
		\addplot+table[x=States, y=GE, col sep=tab]
		{content/Experiments/graphdata/set32.csv};
%		\addlegendentry{\textbf{Greedy Extend}}
	\end{axis}
\end{tikzpicture} 
\label{fig:dataset32}
\end{minipage}% 
&
\begin{minipage}{\dimexpr0.55\textwidth-2\tabcolsep}
\centering
\textbf{Dataset: 8, Symbols: 8}\par\medskip
\begin{tikzpicture}
	\pgfplotsset{every axis legend/.append style={ 
		at={(0.55,1.03)},
		anchor=south}}
	\begin{axis}[
			scaled ticks = true,
			scaled y ticks=base 10:-4,
			xlabel = Number of states,
            	ylabel = Log likelihood,
            	legend columns=-1,
            	legend entries={BW, SBW, GE},
			legend style={/tikz/every even column/.append style={column sep=0.3cm}}]
			
		\addplot+table[x=States, y=BW, col sep=tab]
		{content/Experiments/graphdata/set8.csv};
%		\addlegendentry{\textbf{BW}}
		
		\addplot+table[x=States, y=SBW, col sep=tab]
		{content/Experiments/graphdata/set8.csv};
%		\addlegendentry{\textbf{Sparse BW}}
		
		\addplot+table[x=States, y=GE, col sep=tab]
		{content/Experiments/graphdata/set8.csv};
%		\addlegendentry{\textbf{Greedy Extend}}
	\end{axis}
\end{tikzpicture} 
\label{fig:dataset8}
\end{minipage}
\\
\begin{minipage}[t]{\dimexpr0.6\textwidth-2\tabcolsep}
\end{minipage}
&
\begin{minipage}[t]{\dimexpr0.4\textwidth-2\tabcolsep}
\end{minipage}
\end{tabular*}%
\\
\begin{tabular*}{\textwidth}{@{}cc@{}}
\begin{minipage}{\dimexpr0.55\textwidth-2\tabcolsep}
\centering
\textbf{Dataset: 10, Symbols: 11}\par\medskip
\begin{tikzpicture}
	\pgfplotsset{every axis legend/.append style={ 
		at={(0.55,1.03)},
		anchor=south}}
	\begin{axis}[
			scaled ticks = true,
			scaled y ticks=base 10:-4,
			xlabel = Number of states,
            	ylabel = Log likelihood,
            	legend columns=-1,
            	legend entries={BW, SBW, GE},
			legend style={/tikz/every even column/.append style={column sep=0.3cm}}]
			
		\addplot+table[x=States, y=BW, col sep=tab]
		{content/Experiments/graphdata/set10.csv};
%		\addlegendentry{\textbf{BW}}
		
		\addplot+table[x=States, y=SBW, col sep=tab]
		{content/Experiments/graphdata/set10.csv};
%		\addlegendentry{\textbf{Sparse BW}}
		
		\addplot+table[x=States, y=GE, col sep=tab]
		{content/Experiments/graphdata/set10.csv};
%		\addlegendentry{\textbf{Greedy Extend}}
	\end{axis}
\end{tikzpicture} 
\label{fig:dataset41}
\end{minipage}% 
&
\begin{minipage}{\dimexpr0.55\textwidth-2\tabcolsep}
\centering
\textbf{Dataset: 35, Symbols: 20}\par\medskip
\begin{tikzpicture}
	\pgfplotsset{every axis legend/.append style={ 
		at={(0.55,1.03)},
		anchor=south}}
	\begin{axis}[
			scaled ticks = true,
			scaled y ticks=base 10:-4,
			xlabel = Number of states,
            	ylabel = Log likelihood,
            	legend columns=-1,
            	legend entries={BW, SBW, GE},
			legend style={/tikz/every even column/.append style={column sep=0.3cm}}]
			
		\addplot+table[x=States, y=BW, col sep=tab]
		{content/Experiments/graphdata/set35.csv};
%		\addlegendentry{\textbf{BW}}
		
		\addplot+table[x=States, y=SBW, col sep=tab]
		{content/Experiments/graphdata/set35.csv};
%		\addlegendentry{\textbf{Sparse BW}}
		
		\addplot+table[x=States, y=GE, col sep=tab]
		{content/Experiments/graphdata/set35.csv};
%		\addlegendentry{\textbf{Greedy Extend}}
	\end{axis}
\end{tikzpicture}
\label{fig:dataset35}
\end{minipage}
\\
\begin{minipage}[t]{\dimexpr0.6\textwidth-2\tabcolsep}
\end{minipage}
&
\begin{minipage}[t]{\dimexpr0.4\textwidth-2\tabcolsep}
\end{minipage}
\end{tabular*}%
	
Initial observations
\begin{itemize}
\item BW performs the best on 3 out of 4 sets
\item 
\item 
\end{itemize}	
	
%\subsection{Greedy Extend Experiments}
	%The first big question about the Greedy Extend algorithm, is how the choice of $\beta$ affects the performance of the algorithm.
As $\beta$ denotes the number of Baum Welch iterations to run each time the algorithm attempts to extend the graph, increasing $\beta$ will also increase the run time of the algorithm. It may be the case that better results are achieved when $\beta$ is increased, since more iterations of Baum Welch also means a greater increase in likelihood. However, it could be the case that using many iterations early increases the chance of getting trapped in a local optimum.
An experiment has been conducted of using different values for $\beta$ on data set $1$ from the Pautomac competition. The results can be seen in figure \ref{fig:ge-different-thresholds-tested}. Each line represents the mean value of 5 runs of the Greedy Extend algorithm with the specified number of iterations. Some of the plots have been cut off at a point where one of the runs did not manage to extend beyond a certain number of states.

\begin{figure}
\begin{centering}
\begin{tikzpicture}
	\pgfplotsset{every axis legend/.append style={ 
		at={(0.5,1.03)},
		anchor=south}}
	\begin{axis}[
			xlabel = Number of states,
            	ylabel = Score (lower is better),
            	legend columns=-1,
            	legend entries={IT-0, IT-1, IT-2, IT-3, IT-5, IT-10, IT-50},
			legend style={/tikz/every even column/.append style={column sep=0.3cm}}]
		
		\addplot+[mark=none]table[x=States, y=IT-0, col sep=tab]
		{content/Experiments/graphdata/ge-intermediate-iterations-test.csv};
		
		\addplot+[mark=none]table[x=States, y=IT-1, col sep=tab]
		{content/Experiments/graphdata/ge-intermediate-iterations-test.csv};
		
		\addplot+[mark=none]table[x=States, y=IT-2, col sep=tab]
		{content/Experiments/graphdata/ge-intermediate-iterations-test.csv};
		
		\addplot+[mark=none]table[x=States, y=IT-3, col sep=tab]
		{content/Experiments/graphdata/ge-intermediate-iterations-test.csv};

		\addplot+[mark=none]table[x=States, y=IT-5, col sep=tab]
		{content/Experiments/graphdata/ge-intermediate-iterations-test.csv};
		
		\addplot+[mark=none]table[x=States, y=IT-10, col sep=tab]
		{content/Experiments/graphdata/ge-intermediate-iterations-test.csv};
		
		\addplot+[mark=none]table[x=States, y=IT-50, col sep=tab]
		{content/Experiments/graphdata/ge-intermediate-iterations-test.csv};
	\end{axis}
\end{tikzpicture} 
\caption{Test of different values for $\beta$ while running the Greedy Extend algorithm.}
\label{fig:ge-different-thresholds-tested} 
\end{centering}
\end{figure}

The figure shows surprisingly that using no iterations is somehow better than using just a single iteration. However, using 5 or more iterations is better than not using any iterations at all.
When using 5 iterations, only a single run did not reach 50 states (it stopped at 49).
We choose to conduct further experiments with $\beta = 10$, since all runs with 10 iterations reached 50 states, and it looks like the performance does not change significantly when increasing the number of iterations beyond 10.
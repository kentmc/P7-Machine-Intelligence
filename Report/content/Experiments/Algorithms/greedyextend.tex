\subsubsection{Greedy Extend}
Initially, a graph representation $G$ of a single state \gls{hmm} is created. The single node has initial probability 1, loops to itself with probability 1, and its emission probabilities for each of the $s$ symbols are chosen randomly and normalised.

The following pseudo code describes how the algorithm continuously tries to extend the graph, as long as it improves the likelihood of the data:

\begin{itemize}
\item (1) Repeat $\alpha$ times:
	\begin{itemize}
	\item $G'$ = $(V(G) \cup \{y'\}, E(G))$, where $y'$ is a new node with a random initial probability in range $[0, 1]$ having random emission probabilities for all $s$ symbols, which sums to $1$.
	\item Randomly choose a set of nodes $Y = \{y_1, y_2, ... , y_l\}$ from $V(G')$, where $l = \lceil \log |V(G')| \rceil$ and $\forall a,b: y_a \neq y_b$.
	\item For each $y \in Y$, the transitions $(y, y')$ and $(y', y)$ are added to $E(G')$ with random transition probabilities.
	\item Normalize $G'$.
	\item If $LL(BW^{\beta}(G', D)) > LL(G)$, let $G = LL(BW^{\beta}(G', D))$, go to (1).
	\end{itemize}
\end{itemize}

Return $BW_t(G, D)$.

The value $\alpha$ determines how many times the algorithm will try to improve the model. If it in $\alpha$ tries cannot improve the model further, it stops and returns the model.
$\beta$ determines the number of iterations of the Baum Welch algorithm to run after having extended the model, before it is checked whether adding the node caused an improvement.
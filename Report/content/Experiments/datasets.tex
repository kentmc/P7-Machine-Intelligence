\section{Selecting Datasets}\label{sec:datasets}
In the PautomaC competition, 48 data sets were used. If we were to conduct experiments on all of those data sets, it would take an unreasonable amount of time.
As a consequence, we have chosen to limit the amount of data sets used in our experiments.
We think the most important parameters that differs between data sets are the number of states used by the generating model, the number of symbols, and the transition density. The number of states and symbols have been published after the competition finished, and the transition density is something we can calculate by examining the models that generated each data set, which was also published when the competition finished.
We define the transition density as the ratio between the number of transitions in the model and the number of states squared (the total number of possible transitions). 

For each of the 3 parameters, 4 data sets have been selected that differ as much as possible on that particular parameter, while the other two parameters differ as little as possible.
Figure \ref{fig:statesetplot} shows how four data sets represent different amount of states, while the number of symbols and transition density almost stay the same. In \ref{density_table} and \ref{symbol_table}, a smilar scatterplot can be seen for the transition density and the number of symbols, respectively.

\begin{figure}
	\centering
	\begin{subfigure}[b]{0.5\textwidth}
        	\begin{tikzpicture}
			\begin{axis}[
			scale = 0.8,
			xlabel = Transition density (\%),
            	ylabel = Number of symbols]
			\addplot[scatter,
				only marks,
				scatter src=explicit] 
				table[meta=mark, x=density, y=symbols, col sep=tab]
				{content/Experiments/graphdata/stateset.csv};
			\end{axis}
		\end{tikzpicture}
        \end{subfigure}%
		\begin{subfigure}[b]{0.5\textwidth}
\begin{tikzpicture}
	\begin{axis}[
	scale = 0.8,
			xlabel = Number of states,
            	ylabel = Number of symbols]
		\addplot[scatter,
			only marks,
			scatter src=explicit] 
		table[meta=mark, x=states, y=symbols, col sep=tab]
		{content/Experiments/graphdata/stateset.csv};
	\end{axis}
\end{tikzpicture}
	\end{subfigure}
  	\caption{Two plots visualising the selected state-range datasets}\label{fig:statesetplot}
\end{figure}

%%%%%%%%%%%%%%%%%%%%%%%
%
%_______density range______________

\begin{figure}
	\centering
	\begin{subfigure}[b]{0.5\textwidth}
        	\begin{tikzpicture}
			\begin{axis}[
			scale = 0.8,
			xlabel = Number of states,
            	ylabel = Number of symbols]
			\addplot[scatter,
				only marks,
				scatter src=explicit] 
				table[meta=mark, x=states, y=symbols, col sep=tab]
				{content/Experiments/graphdata/densityset.csv};
			\end{axis}
		\end{tikzpicture}
		\end{subfigure}%
		\begin{subfigure}[b]{0.5\textwidth}
		\begin{tikzpicture}
	\begin{axis}[
			scale = 0.8,
			xlabel = Transition density(\%),
            	ylabel = Number of symbols]
		\addplot[scatter,
			only marks,
			scatter src=explicit] 
		table[meta=mark, x=density, y=symbols, col sep=tab]
		{content/Experiments/graphdata/densityset.csv};
	\end{axis}
\end{tikzpicture}
\end{subfigure}
  	\caption{Two plots visualising the selected density-range datasets}\label{fig:densitysetplot}
\end{figure}

%%%%%%%%%%%%%%%%%%%%%%%
%
%_______symbol range______________

\begin{figure}
	\centering
	\begin{subfigure}[b]{0.5\textwidth}
        	\begin{tikzpicture}
			\begin{axis}[
			scale = 0.8,
			xlabel = Transition density(\%),
            	ylabel = Number of states]
			\addplot[scatter,
				only marks,
				scatter src=explicit] 
				table[meta=mark, x=density, y=states, col sep=tab]
				{content/Experiments/graphdata/symbolset.csv};
			\end{axis}
		\end{tikzpicture}
       \end{subfigure}%
	\begin{subfigure}[b]{0.5\textwidth}
		\begin{tikzpicture}
			\begin{axis}[
			scale = 0.8,
			xlabel = Number of symbols,
            	ylabel = Number of states]
		\addplot[scatter,
			only marks,
			scatter src=explicit] 
		table[meta=mark, x=symbols, y=states, col sep=tab]
		{content/Experiments/graphdata/symbolset.csv};
	\end{axis}
	\end{tikzpicture}
	\end{subfigure}
  	\caption{Two plots visualising the selected symbol-range datasets}\label{fig:symbolsetplot}
\end{figure}

\FloatBarrier

\begin{table}
\centering
{
\begin{tabular}{| c | c | c | c |}
  \hline
  Dataset 	& \textbf{States} 	& Density (\%) 	& Symbols \\  \hline
  6 			& \textbf{19 }			&	13.5				& 6 \\
  23 			& \textbf{33} 			&	11.4				& 7 \\
  41			& \textbf{54} 			&	14.3				& 7 \\
  1				& \textbf{64} 			&	8.7				& 8 \\ \hline
\end{tabular}
\caption{State amount}
\label{state_table}
}
\end{table}

\begin{table}
\centering
{
\begin{tabular}{| c | c | c | c |}
  \hline
  Dataset 	& States  			& \textbf{Density (\%)} 		& Symbols \\  \hline
  36 			&	54					& \textbf{7.4 }						& 9 \\
  8 			&	49					& \textbf{16.8 }					& 8 \\
  43			&	67					& \textbf{40.2 }					& 5 \\
  37			&	69					& \textbf{54 	}					& 8 \\ \hline
\end{tabular}
\caption{Transition density percentage}
\label{density_table}
}
\end{table}

\begin{table}
\centering
{
\begin{tabular}{| c | c | c | c |}
  \hline
  Dataset 	& States	 	& Density (\%) 	& \textbf{Symbols}	 \\  \hline
  32 			& 43				& 11.8				& \textbf{4}		\\
  8 			& 49				& 16.8 				& \textbf{8}		\\
  10			& 49				& 14.2 				& \textbf{11}	\\
  35			& 47				& 14.2 				& \textbf{20}	\\ \hline
\end{tabular}
\caption{Symbol alphabet size}
\label{symbol_table}

}
\end{table}
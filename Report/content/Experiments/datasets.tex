\section{Datasets}
Pautomac delivered 48 datasets, which given this project's time constraint is too many, so a dataset selection method has to be devised. We want to understand how the presented algorithms handle the different known variations, of the model that originally created the datasets. There are three variables; the amount of states, a density percentage and the observable symbol alphabet size.

There is no obvious pattern in which Pautomac produced these datasets, which can easily be seen when plotting the datasets in a coordinate system:

\begin{filecontents*}{data.csv}
density,symbols,mark
14.30041152,7,a
8.717561099,8,a
7.475994513,9,a
2.168367347,6,a
4.283194759,4,a
3.503460208,7,a
16.86797168,8,a
12.11995002,10,a
4.710076938,6,a
2.182106725,4,a
14.24406497,11,a
7.084910763,4,a
12.68639053,14,a
4.5625,10,a
8.179012346,6,a
5.610808782,13,a
7.229239452,15,a
11.47842057,7,a
40.18712408,5,a
1.662887377,18,a
11.34215501,6,a
23.04,4,a
13.5734072,6,a
14.21457673,20,a
19.23947488,20,a
19.75097656,21,a
54.0852762,8,a
32.69230769,14,a
26.24793388,21,a
19.3877551,10,a
8,12,a
22.91666667,4,a
4.974489796,23,a
38.2231405,13,a
23.61111111,13,a
11.83431953,15,a
16.96,20,a
34.72222222,5,a
38.88888889,13,a
8.673469388,19,a
36.11111111,5,a
15.70247934,18,a
44.59833795,17,a
61.11111111,14,a
66.66666667,9,a
69.13580247,10,a
63.28125,23,a
67.31301939,23,a
\end{filecontents*}

\begin{tikzpicture}
	\begin{axis}[
			scatter/use mapped color={draw=blue,fill=blue!30!white}, 
			only marks,
			xlabel = Transition density (\%),
            	ylabel = Number of symbols]
		\addplot+table[x=density, y=symbols, col sep=comma, mark options={blue}]
		{data.csv};
	\end{axis}
\end{tikzpicture}

The range of the three variables is known, where we want to isolate each variable, to determine how the presented algorithms behave. There will be no optimal set of datasets, for isolating one variable, but by calculating the euclidean distance between a preselected point and all datasets, it is possible to select the best-case.

The preselected point has to be explored. A natural first step is to determine the two static values based on a dense cluster in their own two-dimensional space. Having determined a plot for the two static values, only the variable value points has to determined, so that the range of this variable is covered sufficiently. For instance, selecting the datasets for the state range we use the above figure, where a good cluster can be found at point x/y. Then the range is covered from 10 to 70 by 10 steps. These 7 steps were close to only 4 datasets.  

This process of selecting datasets for one variable range at a time, were applied for all three variables:


\begin{table}
\centering
{
\textbf{State amount}

\begin{tabular}{|c|c|}
  \hline
  Dataset 	& States \\  \hline
  5 			& 19 \\
  23 			& 33 \\
  41			& 54 \\
  1				& 64 \\ \hline
\end{tabular}

\textbf{Transition density percentage}

\begin{tabular}{|c|c|}
  \hline
  Dataset 	& Percentage \\  \hline
  36 			& 7.4 \\
  8 			& 16.8 \\
  43			& 40.2 \\
  37			& 54 \\ \hline
\end{tabular}


\textbf{Symbol alphabet size}

\begin{tabular}{|c|c|}
  \hline
  Dataset 	& Symbols \\  \hline
  32 			& 4 \\
  8 			& 8 \\
  10			& 11 \\
  35			& 20 \\ \hline
\end{tabular}
}
\end{table}
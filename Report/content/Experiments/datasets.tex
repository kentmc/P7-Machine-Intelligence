\section{Selecting Datasets}
Pautomac delivered 48 datasets, which is more than what is necessary to determine how the different learners behave on the different parameters for the datasets. Three parameters are known about the datasets: the amount of states that was used in the model that created the data, the amount of different symbols emitted by the model and the transition density of the model. One approach is of course to run experiments on all the datasets and accumulate the results. However using a representative set of datasets for each parameter variable, will produce the necessary results as well. The chosen approach is to select four representative datasets for each parameter variable. There is no perfect set of datasets, however the best fitting datasets have been chosen.
Figure \ref{fig:statesetplot} shows how four datasets represents the range of the state amount parameter variable. The left scatterplot shows the datasets plotted by transition density in percentage and number of emission symbols, where the selected datasets are clustered together. The right scatterplot shows how the datasets has a much wider spread, following the number of state axis. This process of selecting nicely fitting datasets for each parameter range were applied to all three variables. The scatterplots for each variable can be seen in Figure \ref{fig:statesetplot}, \ref{fig:densitysetplot} and \ref{fig:symbolsetplot}. The  selected datasets with all their parameters can be seen in table \ref{state_table}, \ref{density_table} and \ref{symbol_table}.

\begin{figure}
	\centering
	\begin{subfigure}[b]{0.5\textwidth}
        	\begin{tikzpicture}
			\begin{axis}[
			scale = 0.8,
			xlabel = Transition density (\%),
            	ylabel = Number of symbols]
			\addplot[scatter,
				only marks,
				scatter src=explicit] 
				table[meta=mark, x=density, y=symbols, col sep=tab]
				{content/Experiments/graphdata/stateset.csv};
			\end{axis}
		\end{tikzpicture}
        \end{subfigure}%
		\begin{subfigure}[b]{0.5\textwidth}
\begin{tikzpicture}
	\begin{axis}[
	scale = 0.8,
			xlabel = Number of states,
            	ylabel = Number of symbols]
		\addplot[scatter,
			only marks,
			scatter src=explicit] 
		table[meta=mark, x=states, y=symbols, col sep=tab]
		{content/Experiments/graphdata/stateset.csv};
	\end{axis}
\end{tikzpicture}
	\end{subfigure}
  	\caption{Two plots visualising the selected state-range datasets}\label{fig:statesetplot}
\end{figure}

%%%%%%%%%%%%%%%%%%%%%%%
%
%_______density range______________

\begin{figure}
	\centering
	\begin{subfigure}[b]{0.5\textwidth}
        	\begin{tikzpicture}
			\begin{axis}[
			scale = 0.8,
			xlabel = Number of states,
            	ylabel = Number of symbols]
			\addplot[scatter,
				only marks,
				scatter src=explicit] 
				table[meta=mark, x=states, y=symbols, col sep=tab]
				{content/Experiments/graphdata/densityset.csv};
			\end{axis}
		\end{tikzpicture}
		\end{subfigure}%
		\begin{subfigure}[b]{0.5\textwidth}
		\begin{tikzpicture}
	\begin{axis}[
			scale = 0.8,
			xlabel = Transition density(\%),
            	ylabel = Number of symbols]
		\addplot[scatter,
			only marks,
			scatter src=explicit] 
		table[meta=mark, x=density, y=symbols, col sep=tab]
		{content/Experiments/graphdata/densityset.csv};
	\end{axis}
\end{tikzpicture}
\end{subfigure}
  	\caption{Two plots visualising the selected density-range datasets}\label{fig:densitysetplot}
\end{figure}

%%%%%%%%%%%%%%%%%%%%%%%
%
%_______symbol range______________

\begin{figure}
	\centering
	\begin{subfigure}[b]{0.5\textwidth}
        	\begin{tikzpicture}
			\begin{axis}[
			scale = 0.8,
			xlabel = Transition density(\%),
            	ylabel = Number of states]
			\addplot[scatter,
				only marks,
				scatter src=explicit] 
				table[meta=mark, x=density, y=states, col sep=tab]
				{content/Experiments/graphdata/symbolset.csv};
			\end{axis}
		\end{tikzpicture}
       \end{subfigure}%
	\begin{subfigure}[b]{0.5\textwidth}
		\begin{tikzpicture}
			\begin{axis}[
			scale = 0.8,
			xlabel = Number of symbols,
            	ylabel = Number of states]
		\addplot[scatter,
			only marks,
			scatter src=explicit] 
		table[meta=mark, x=symbols, y=states, col sep=tab]
		{content/Experiments/graphdata/symbolset.csv};
	\end{axis}
	\end{tikzpicture}
	\end{subfigure}
  	\caption{Two plots visualising the selected symbol-range datasets}\label{fig:symbolsetplot}
\end{figure}

\FloatBarrier

\begin{table}
\centering
{
\begin{tabular}{| c | c | c | c |}
  \hline
  Dataset 	& \textbf{States} 	& Density (\%) 	& Symbols \\  \hline
  6 			& \textbf{19 }			&	13.5				& 6 \\
  23 			& \textbf{33} 			&	11.4				& 7 \\
  41			& \textbf{54} 			&	14.3				& 7 \\
  1				& \textbf{64} 			&	8.7				& 8 \\ \hline
\end{tabular}
\caption{State amount}
\label{state_table}
}
\end{table}

\begin{table}
\centering
{
\begin{tabular}{| c | c | c | c |}
  \hline
  Dataset 	& States  			& \textbf{Density (\%)} 		& Symbols \\  \hline
  36 			&	54					& \textbf{7.4 }						& 9 \\
  8 			&	49					& \textbf{16.8 }					& 7 \\
  43			&	67					& \textbf{40.2 }					& 5 \\
  37			&	69					& \textbf{54 	}					& 8 \\ \hline
\end{tabular}
\caption{Transition density percentage}
\label{density_table}
}
\end{table}

\begin{table}
\centering
{
\begin{tabular}{| c | c | c | c |}
  \hline
  Dataset 	& States	 	& Density (\%) 	& \textbf{Symbols}	 \\  \hline
  32 			& 43				& 11.8				& \textbf{4}		\\
  8 			& 49				& 16.7 				& \textbf{8}		\\
  10			& 49				& 14.2 				& \textbf{11}	\\
  35			& 47				& 14.2 				& \textbf{20}	\\ \hline
\end{tabular}
\caption{Symbol alphabet size}
\label{symbol_table}

}
\end{table}
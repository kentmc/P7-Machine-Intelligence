\section{Selecting Datasets}
Pautomac delivered 48 datasets, which given this project's time constraint is too many, so a dataset selection method has to be devised. We want to understand how the presented algorithms handle the different known variations, of the model that originally created the datasets. There are three variables; the amount of states, a density percentage and the observable symbol alphabet size.

There is no obvious pattern in which Pautomac produced these datasets, which can easily be seen when plotting the datasets in a coordinate system:

\begin{tikzpicture}
	\begin{axis}[
			scatter/use mapped color={draw=blue,fill=blue!30!white}, 
			only marks,
			xlabel = Transition density (\%),
            	ylabel = Number of symbols]
		\addplot+table[x=density, y=symbols, col sep=comma, mark options={blue}]
		{content/Experiments/stateset.csv};
	\end{axis}
\end{tikzpicture}

The range of the three variables is known, where we want to isolate each variable one at a time, to determine how the presented algorithms behave. There will be no optimal set of datasets, for isolating one variable, but by calculating the euclidean distance between a preselected point and all the datasets, it is possible to select the best-case.

The preselected point, has to be found by exploration. A natural first step is to determine the two static values based on a dense cluster in their own two-dimensional space. Having determined a plot for the two static values, some points for the variable have to be determined, so that the range of this variable is covered sufficiently. For instance, selecting the datasets for the state range we can use the above figure, where a cluster can be found at bottom left area. A point in this cluster is selected, determining the two static values and then the range of the variable is covered from 10 to 70 by 10 steps. Of course, not all of the points in the cluster will be close in the third dimension, so the 7 steps of the variable value, only came close to 4 datasets.

This process of selecting the best-case datasets for  at a time, were applied for all three variables:

\begin{table}
\centering
{
\textbf{State amount}

\begin{tabular}{| c | c | c | c |}
  \hline
  Dataset 	& \textbf{States} 	& Density (\%) 	& Symbols \\  \hline
  6 			& \textbf{19 }			&	13.5				& 6 \\
  23 			& \textbf{33} 			&	11.4				& 7 \\
  41			& \textbf{54} 			&	14.3				& 7 \\
  1				& \textbf{64} 			&	8.7				& 8 \\ \hline
\end{tabular}

\textbf{Transition density percentage}

\begin{tabular}{| c | c | c | c |}
  \hline
  Dataset 	& States  			& \textbf{Density (\%)} 		& Symbols \\  \hline
  36 			&	54					& \textbf{7.4 }						& 9 \\
  8 			&	49					& \textbf{16.8 }					& 7 \\
  43			&	67					& \textbf{40.2 }					& 5 \\
  37			&	69					& \textbf{54 	}					& 8 \\ \hline
\end{tabular}


\textbf{Symbol alphabet size}

\begin{tabular}{| c | c | c | c |}
  \hline
  Dataset 	& States	 	& Density (\%) 	& \textbf{Symbols}	 \\  \hline
  32 			& 43				& 11.8				& \textbf{4}		\\
  8 			& 49				& 16.7 				& \textbf{8}		\\
  10			& 49				& 14.2 				& \textbf{11}	\\
  35			& 47				& 14.2 				& \textbf{20}	\\ \hline
\end{tabular}

}
\end{table}
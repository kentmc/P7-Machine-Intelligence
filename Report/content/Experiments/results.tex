\FloatBarrier

\section{Experiments}\label{sec:results}

\subsection{Dataset Parameter Experiment}
The approach used for each dataset parameter experiment have been to cover a range of state amounts, ranging from 10 to 100 states, together with the previously define parameters. \gls{bw} was probed on specific state amounts with step sizes of 10, where the dynamic algorithms have been defined to reach either some local maxima or stop at 100 states.

\paragraph{State Space Experiments}
The purpose of the state experiments are to show how \gls{bw} and our own algorithms and models perform in relation to the state space.

	\begin{figure}[h]
\begin{tabular*}{\textwidth}{@{}cc@{}}
\begin{minipage}{\dimexpr0.55\textwidth-2\tabcolsep}
\centering
\textbf{Data set: 6, States: 19}\par\medskip
\begin{tikzpicture}
	\pgfplotsset{every axis legend/.append style={ 
		at={(0.5,1.1)},
		anchor=south}}
	\begin{axis}[
			scaled ticks = true,
			scaled y ticks=base 10:-4,
			xlabel = Number of states,
            	ylabel = Log likelihood,
            	legend columns=-1,
            	legend entries={BW, SBW, GE, GS},
			legend style={/tikz/every even column/.append style={column sep=0.3cm}}]
			
		\addplot+table[x=States, y=BW, col sep=tab]
		{content/Experiments/graphdata/set6.csv};
%		\addlegendentry{\textbf{BW}}
		
		\addplot+table[x=States, y=SBW, col sep=tab]
		{content/Experiments/graphdata/set6.csv};
%		\addlegendentry{\textbf{Sparse BW}}
		
		\addplot+table[x=States, y=GE, col sep=tab]
		{content/Experiments/graphdata/set6.csv};
		
		\addplot+table[x=States, y=Padawan, col sep=tab]
		{content/Experiments/graphdata/set6.csv};
%		\addlegendentry{\textbf{Greedy Extend}}
	\end{axis}
\end{tikzpicture} 
\end{minipage}% 
&
\begin{minipage}{\dimexpr0.55\textwidth-2\tabcolsep}
\centering
\textbf{Data set: 23, States: 33}\par\medskip
\begin{tikzpicture}
	\pgfplotsset{every axis legend/.append style={ 
		at={(0.5,1.1)},
		anchor=south}}
	\begin{axis}[
			scaled ticks = true,
			scaled y ticks=base 10:-4,
			xlabel = Number of states,
            	ylabel = Score (lower is better),
            	legend columns=-1,
            	legend entries={BW, SBW, GE, GS},
			legend style={/tikz/every even column/.append style={column sep=0.3cm}}]
			
		\addplot+table[x=States, y=BW, col sep=tab]
		{content/Experiments/graphdata/set23.csv};
%		\addlegendentry{\textbf{BW}}
		
		\addplot+table[x=States, y=SBW, col sep=tab]
		{content/Experiments/graphdata/set23.csv};
%		\addlegendentry{\textbf{Sparse BW}}
		
		\addplot+table[x=States, y=GE, col sep=tab]
		{content/Experiments/graphdata/set23.csv};
		
		\addplot+table[x=States, y=Padawan, col sep=tab]
		{content/Experiments/graphdata/set23.csv};
%		\addlegendentry{\textbf{Greedy Extend}}
	\end{axis}
\end{tikzpicture} 
\end{minipage}
\\
\begin{minipage}[t]{\dimexpr0.6\textwidth-2\tabcolsep}
\end{minipage}
&
\begin{minipage}[t]{\dimexpr0.4\textwidth-2\tabcolsep}
\end{minipage}
\end{tabular*}%
\\
\begin{tabular*}{\textwidth}{@{}cc@{}}
\begin{minipage}{\dimexpr0.55\textwidth-2\tabcolsep}
\centering
\textbf{Data set: 41, States: 54}\par\medskip
\begin{tikzpicture}
	\pgfplotsset{every axis legend/.append style={ 
		at={(0.5,1.1)},
		anchor=south}}
	\begin{axis}[
			scaled ticks = true,
			scaled y ticks=base 10:-4,
			xlabel = Number of states,
            	ylabel = Score (lower is better),
            	legend columns=-1,
            	legend entries={BW, SBW, GE, GS},
			legend style={/tikz/every even column/.append style={column sep=0.3cm}}]
			
		\addplot+table[x=States, y=BW, col sep=tab]
		{content/Experiments/graphdata/set41.csv};
%		\addlegendentry{\textbf{BW}}
		
		\addplot+table[x=States, y=SBW, col sep=tab]
		{content/Experiments/graphdata/set41.csv};
%		\addlegendentry{\textbf{Sparse BW}}
		
		\addplot+table[x=States, y=GE, col sep=tab]
		{content/Experiments/graphdata/set41.csv};
		
		\addplot+table[x=States, y=Padawan, col sep=tab]
		{content/Experiments/graphdata/set41.csv};
%		\addlegendentry{\textbf{Greedy Extend}}
	\end{axis}
\end{tikzpicture} 
\end{minipage}% 
&
\begin{minipage}{\dimexpr0.55\textwidth-2\tabcolsep}
\centering
\textbf{Data set: 1, States: 64}\par\medskip
\begin{tikzpicture}
	\pgfplotsset{every axis legend/.append style={ 
		at={(0.5,1.1)},
		anchor=south}}
	\begin{axis}[
			scaled ticks = true,
			scaled y ticks=base 10:-4,
			xlabel = Number of states,
            	ylabel = Log likelihood,
            	legend columns=-1,
            	legend entries={BW, SBW, GE, GS},
			legend style={/tikz/every even column/.append style={column sep=0.3cm}}]
			
		\addplot+table[x=States, y=BW, col sep=tab]
		{content/Experiments/graphdata/set1.csv};
%		\addlegendentry{\textbf{BW}}
		
		\addplot+table[x=States, y=SBW, col sep=tab]
		{content/Experiments/graphdata/set1.csv};
%		\addlegendentry{\textbf{Sparse BW}}
		
		\addplot+table[x=States, y=GE, col sep=tab]
		{content/Experiments/graphdata/set1.csv};
		
		\addplot+table[x=States, y=Padawan, col sep=tab]
		{content/Experiments/graphdata/set1.csv};
%		\addlegendentry{\textbf{Greedy Extend}}
	\end{axis}
\end{tikzpicture} 
\end{minipage}
\\
\begin{minipage}[t]{\dimexpr0.6\textwidth-2\tabcolsep}
\end{minipage}
\end{tabular*}%
\caption{Four data sets produced by models with 19 to 64 states}\label{fig:states}
\end{figure}

Initial observations
\begin{itemize}
\item GE performs the best, except dataset 1. It seems like GE would surpass BW with a larger state space.
\item SBW performs worse than BW, note that dataset 41 has a fairly small y-axis, where the best GE result is about 800 points. We estimate an error margin of 100-200 points. Without more runs for each experiment, it is impossible to determine an exact error margin.
\item In general we seem to 
\end{itemize}


\paragraph{Transition Density Experiments}
The purpose of the transition density experiments are to show how \gls{bw} and our own algorithms and models perform in relation to the transition density.

	\FloatBarrier
\section{Results}
\subsection{State Experiments}

\begin{tabular*}{\textwidth}{@{}cc@{}}
\begin{minipage}{\dimexpr0.55\textwidth-2\tabcolsep}
\centering
\textbf{Dataset: 6, States: 19}\par\medskip
\begin{tikzpicture}
	\begin{axis}[
			xlabel = Number of states,
            	ylabel = Score (lower is better)]
		\addplot+table[x=States, y=BW, col sep=tab]
		{content/Experiments/graphdata/set6.csv};
		\addlegendentry{\textbf{BW}}
		
		\addplot+table[x=States, y=SBW, col sep=tab]
		{content/Experiments/graphdata/set6.csv};
		\addlegendentry{\textbf{Sparse BW}}
		
		\addplot+table[x=States, y=GE, col sep=tab]
		{content/Experiments/graphdata/set6.csv};
		\addlegendentry{\textbf{Greedy Extend}}
	\end{axis}
\end{tikzpicture} 
\label{fig:dataset6}
\end{minipage}% 
&
\begin{minipage}{\dimexpr0.55\textwidth-2\tabcolsep}
\centering
\textbf{Dataset: 23, States: 33}\par\medskip
\begin{tikzpicture}
	\begin{axis}[
			xlabel = Number of states,
            	ylabel = Score (lower is better)]
		\addplot+table[x=States, y=BW, col sep=tab]
		{content/Experiments/graphdata/set23.csv};
		\addlegendentry{\textbf{BW}}
		
		\addplot+table[x=States, y=SBW, col sep=tab]
		{content/Experiments/graphdata/set23.csv};
		\addlegendentry{\textbf{Sparse BW}}
		
		\addplot+table[x=States, y=GE, col sep=tab]
		{content/Experiments/graphdata/set23.csv};
		\addlegendentry{\textbf{Greedy Extend}}
	\end{axis}
\end{tikzpicture} 
\label{fig:dataset23}
\end{minipage}
\\
\begin{minipage}[t]{\dimexpr0.6\textwidth-2\tabcolsep}
\end{minipage}
&
\begin{minipage}[t]{\dimexpr0.4\textwidth-2\tabcolsep}
\end{minipage}
\end{tabular*}%


begin{tabular*}{\textwidth}{@{}cc@{}}
\begin{minipage}{\dimexpr0.55\textwidth-2\tabcolsep}
\centering
\textbf{Dataset: 41, States: 54}\par\medskip
\begin{tikzpicture}
	\begin{axis}[
			xlabel = Number of states,
            	ylabel = Score (lower is better)]
		\addplot+table[x=States, y=BW, col sep=tab]
		{content/Experiments/graphdata/set41.csv};
		\addlegendentry{\textbf{BW}}
		
		\addplot+table[x=States, y=SBW, col sep=tab]
		{content/Experiments/graphdata/set41.csv};
		\addlegendentry{\textbf{Sparse BW}}
		
		\addplot+table[x=States, y=GE, col sep=tab]
		{content/Experiments/graphdata/set41.csv};
		\addlegendentry{\textbf{Greedy Extend}}
	\end{axis}
\end{tikzpicture} 
\label{fig:dataset41}
\end{minipage}% 
&
\begin{minipage}{\dimexpr0.55\textwidth-2\tabcolsep}
\centering
\textbf{Dataset: 1, States: 64}\par\medskip
\begin{tikzpicture}
	\begin{axis}[
			xlabel = Number of states,
            	ylabel = Score (lower is better)]
		\addplot+table[x=States, y=BW, col sep=tab]
		{content/Experiments/graphdata/set64.csv};
		\addlegendentry{\textbf{BW}}
		
		\addplot+table[x=States, y=SBW, col sep=tab]
		{content/Experiments/graphdata/set64.csv};
		\addlegendentry{\textbf{Sparse BW}}
		
		\addplot+table[x=States, y=GE, col sep=tab]
		{content/Experiments/graphdata/set64.csv};
		\addlegendentry{\textbf{Greedy Extend}}
	\end{axis}
\end{tikzpicture} 
\label{fig:dataset64}
\end{minipage}
\\
\begin{minipage}[t]{\dimexpr0.6\textwidth-2\tabcolsep}
\end{minipage}
&
\begin{minipage}[t]{\dimexpr0.4\textwidth-2\tabcolsep}
\end{minipage}
\end{tabular*}%

\subsection{Density Experiments}

\FloatBarrier
\begin{tabular*}{\textwidth}{@{}cc@{}}
\begin{minipage}{\dimexpr0.55\textwidth-2\tabcolsep}
\centering
\begin{tikzpicture}
	\begin{axis}[
			xlabel = Number of states,
            	ylabel = Score (lower is better)]
		\addplot+table[x=States, y=BW, col sep=tab]
		{content/Experiments/graphdata/set23.csv};
		\addlegendentry{\textbf{BW}}
		
		\addplot+table[x=States, y=SBW, col sep=tab]
		{content/Experiments/graphdata/set23.csv};
		\addlegendentry{\textbf{Sparse BW}}
		
		\addplot+table[x=States, y=GE, col sep=tab]
		{content/Experiments/graphdata/set23.csv};
		\addlegendentry{\textbf{Greedy Extend}}
		
	\end{axis}
\end{tikzpicture}
\label{dataset23}
\end{minipage}% 
&
\begin{minipage}{\dimexpr0.55\textwidth-2\tabcolsep}
\centering
\begin{tikzpicture}
	\begin{axis}[
			xlabel = Number of states,
            	ylabel = Score (lower is better)]
		\addplot+table[x=States, y=BW, col sep=tab]
		{content/Experiments/graphdata/set1.csv};
		\addlegendentry{\textbf{Dataset: 1 States: 64}}
	\end{axis}
\end{tikzpicture}
\label{dataset1}
\end{minipage}
\\
\begin{minipage}[t]{\dimexpr0.6\textwidth-2\tabcolsep}
\end{minipage}
&
\begin{minipage}[t]{\dimexpr0.4\textwidth-2\tabcolsep}
\end{minipage}
\end{tabular*}%

\begin{tabular*}{\textwidth}{@{}cc@{}}
\begin{minipage}{\dimexpr0.55\textwidth-2\tabcolsep}
\centering
\begin{tikzpicture}
	\begin{axis}[
			xlabel = Number of states,
            	ylabel = Score (lower is better)]
		\addplot+table[x=States, y=BW, col sep=tab]
		{content/Experiments/graphdata/set23.csv};
		\addlegendentry{\textbf{Dataset: 23 States: 33}}
	\end{axis}
\end{tikzpicture}
\label{dataset23}
\end{minipage}% 
&
\begin{minipage}{\dimexpr0.55\textwidth-2\tabcolsep}
\centering
\begin{tikzpicture}
	\begin{axis}[
			xlabel = Number of states,
            	ylabel = Score (lower is better)]
		\addplot+table[x=States, y=BW, col sep=tab]
		{content/Experiments/graphdata/set1.csv};
		\addlegendentry{\textbf{Dataset: 1 States: 64}}
	\end{axis}
\end{tikzpicture}
\label{dataset1}
\end{minipage}
\\
\begin{minipage}[t]{\dimexpr0.6\textwidth-2\tabcolsep}
\end{minipage}
&
\begin{minipage}[t]{\dimexpr0.4\textwidth-2\tabcolsep}
\end{minipage}
\end{tabular*}%

\subsection{Symbol size Experiments}

\begin{tabular*}{\textwidth}{@{}cc@{}}
\begin{minipage}{\dimexpr0.55\textwidth-2\tabcolsep}
\centering
\textbf{Dataset: 6, States: 19}\par\medskip
\begin{tikzpicture}
	\begin{axis}[
			xlabel = Number of states,
            	ylabel = Score (lower is better)]
		\addplot+table[x=States, y=BW, col sep=tab]
		{content/Experiments/graphdata/set6.csv};
		\addlegendentry{\textbf{BW}}
		
		\addplot+table[x=States, y=SBW, col sep=tab]
		{content/Experiments/graphdata/set6.csv};
		\addlegendentry{\textbf{Sparse BW}}
		
		\addplot+table[x=States, y=GE, col sep=tab]
		{content/Experiments/graphdata/set6.csv};
		\addlegendentry{\textbf{Greedy Extend}}
	\end{axis}
\end{tikzpicture} 
\label{fig:dataset6}
\end{minipage}% 
&
\begin{minipage}{\dimexpr0.55\textwidth-2\tabcolsep}
\centering
\textbf{Dataset: 23, States: 33}\par\medskip
\begin{tikzpicture}
	\begin{axis}[
			xlabel = Number of states,
            	ylabel = Score (lower is better)]
		\addplot+table[x=States, y=BW, col sep=tab]
		{content/Experiments/graphdata/set23.csv};
		\addlegendentry{\textbf{BW}}
		
		\addplot+table[x=States, y=SBW, col sep=tab]
		{content/Experiments/graphdata/set23.csv};
		\addlegendentry{\textbf{Sparse BW}}
		
		\addplot+table[x=States, y=GE, col sep=tab]
		{content/Experiments/graphdata/set23.csv};
		\addlegendentry{\textbf{Greedy Extend}}
	\end{axis}
\end{tikzpicture} 
\label{fig:dataset23}
\end{minipage}
\\
\begin{minipage}[t]{\dimexpr0.6\textwidth-2\tabcolsep}
\end{minipage}
&
\begin{minipage}[t]{\dimexpr0.4\textwidth-2\tabcolsep}
\end{minipage}
\end{tabular*}%


begin{tabular*}{\textwidth}{@{}cc@{}}
\begin{minipage}{\dimexpr0.55\textwidth-2\tabcolsep}
\centering
\textbf{Dataset: 41, States: 54}\par\medskip
\begin{tikzpicture}
	\begin{axis}[
			xlabel = Number of states,
            	ylabel = Score (lower is better)]
		\addplot+table[x=States, y=BW, col sep=tab]
		{content/Experiments/graphdata/set41.csv};
		\addlegendentry{\textbf{BW}}
		
		\addplot+table[x=States, y=SBW, col sep=tab]
		{content/Experiments/graphdata/set41.csv};
		\addlegendentry{\textbf{Sparse BW}}
		
		\addplot+table[x=States, y=GE, col sep=tab]
		{content/Experiments/graphdata/set41.csv};
		\addlegendentry{\textbf{Greedy Extend}}
	\end{axis}
\end{tikzpicture} 
\label{fig:dataset41}
\end{minipage}% 
&
\begin{minipage}{\dimexpr0.55\textwidth-2\tabcolsep}
\centering
\textbf{Dataset: 1, States: 64}\par\medskip
\begin{tikzpicture}
	\begin{axis}[
			xlabel = Number of states,
            	ylabel = Score (lower is better)]
		\addplot+table[x=States, y=BW, col sep=tab]
		{content/Experiments/graphdata/set64.csv};
		\addlegendentry{\textbf{BW}}
		
		\addplot+table[x=States, y=SBW, col sep=tab]
		{content/Experiments/graphdata/set64.csv};
		\addlegendentry{\textbf{Sparse BW}}
		
		\addplot+table[x=States, y=GE, col sep=tab]
		{content/Experiments/graphdata/set64.csv};
		\addlegendentry{\textbf{Greedy Extend}}
	\end{axis}
\end{tikzpicture} 
\label{fig:dataset64}
\end{minipage}
\\
\begin{minipage}[t]{\dimexpr0.6\textwidth-2\tabcolsep}
\end{minipage}
&
\begin{minipage}[t]{\dimexpr0.4\textwidth-2\tabcolsep}
\end{minipage}
\end{tabular*}%

Initial observations
\begin{itemize}
\item GE performs the best
\item 
\item 
\end{itemize}

\paragraph{Symbol Alphabet Size Experiments}
The purpose of the s are to show how \gls{bw} and our own algorithms and models perform in relation to the size of the emission symbol alphabet size.

	\begin{tabular*}{\textwidth}{@{}cc@{}}
\begin{minipage}{\dimexpr0.55\textwidth-2\tabcolsep}
\centering
\textbf{Dataset: 32, Symbols: 4}\par\medskip
\begin{tikzpicture}
	\pgfplotsset{every axis legend/.append style={ 
		at={(0.55,1.03)},
		anchor=south}}
	\begin{axis}[
			scaled ticks = true,
			scaled y ticks=base 10:-4,
			xlabel = Number of states,
            	ylabel = Score (lower is better),
            	legend columns=-1,
            	legend entries={BW, SBW, GE},
			legend style={/tikz/every even column/.append style={column sep=0.3cm}}]
			
		\addplot+table[x=States, y=BW, col sep=tab]
		{content/Experiments/graphdata/set32.csv};
%		\addlegendentry{\textbf{BW}}
		
		\addplot+table[x=States, y=SBW, col sep=tab]
		{content/Experiments/graphdata/set32.csv};
%		\addlegendentry{\textbf{Sparse BW}}
		
		\addplot+table[x=States, y=GE, col sep=tab]
		{content/Experiments/graphdata/set32.csv};
%		\addlegendentry{\textbf{Greedy Extend}}
	\end{axis}
\end{tikzpicture} 
\label{fig:dataset32}
\end{minipage}% 
&
\begin{minipage}{\dimexpr0.55\textwidth-2\tabcolsep}
\centering
\textbf{Dataset: 8, Symbols: 8}\par\medskip
\begin{tikzpicture}
	\pgfplotsset{every axis legend/.append style={ 
		at={(0.55,1.03)},
		anchor=south}}
	\begin{axis}[
			scaled ticks = true,
			scaled y ticks=base 10:-4,
			xlabel = Number of states,
            	ylabel = Score (lower is better),
            	legend columns=-1,
            	legend entries={BW, SBW, GE},
			legend style={/tikz/every even column/.append style={column sep=0.3cm}}]
			
		\addplot+table[x=States, y=BW, col sep=tab]
		{content/Experiments/graphdata/set8.csv};
%		\addlegendentry{\textbf{BW}}
		
		\addplot+table[x=States, y=SBW, col sep=tab]
		{content/Experiments/graphdata/set8.csv};
%		\addlegendentry{\textbf{Sparse BW}}
		
		\addplot+table[x=States, y=GE, col sep=tab]
		{content/Experiments/graphdata/set8.csv};
%		\addlegendentry{\textbf{Greedy Extend}}
	\end{axis}
\end{tikzpicture} 
\label{fig:dataset8}
\end{minipage}
\\
\begin{minipage}[t]{\dimexpr0.6\textwidth-2\tabcolsep}
\end{minipage}
&
\begin{minipage}[t]{\dimexpr0.4\textwidth-2\tabcolsep}
\end{minipage}
\end{tabular*}%
\\
\begin{tabular*}{\textwidth}{@{}cc@{}}
\begin{minipage}{\dimexpr0.55\textwidth-2\tabcolsep}
\centering
\textbf{Dataset: 10, Symbols: 11}\par\medskip
\begin{tikzpicture}
	\pgfplotsset{every axis legend/.append style={ 
		at={(0.55,1.03)},
		anchor=south}}
	\begin{axis}[
			scaled ticks = true,
			scaled y ticks=base 10:-4,
			xlabel = Number of states,
            	ylabel = Score (lower is better),
            	legend columns=-1,
            	legend entries={BW, SBW, GE},
			legend style={/tikz/every even column/.append style={column sep=0.3cm}}]
			
		\addplot+table[x=States, y=BW, col sep=tab]
		{content/Experiments/graphdata/set10.csv};
%		\addlegendentry{\textbf{BW}}
		
		\addplot+table[x=States, y=SBW, col sep=tab]
		{content/Experiments/graphdata/set10.csv};
%		\addlegendentry{\textbf{Sparse BW}}
		
		\addplot+table[x=States, y=GE, col sep=tab]
		{content/Experiments/graphdata/set10.csv};
%		\addlegendentry{\textbf{Greedy Extend}}
	\end{axis}
\end{tikzpicture} 
\label{fig:dataset41}
\end{minipage}% 
&
\begin{minipage}{\dimexpr0.55\textwidth-2\tabcolsep}
\centering
\textbf{Dataset: 35, Symbols: 20}\par\medskip
\begin{tikzpicture}
	\pgfplotsset{every axis legend/.append style={ 
		at={(0.55,1.03)},
		anchor=south}}
	\begin{axis}[
			scaled ticks = true,
			scaled y ticks=base 10:-4,
			xlabel = Number of states,
            	ylabel = Score (lower is better),
            	legend columns=-1,
            	legend entries={BW, SBW, GE},
			legend style={/tikz/every even column/.append style={column sep=0.3cm}}]
			
		\addplot+table[x=States, y=BW, col sep=tab]
		{content/Experiments/graphdata/set35.csv};
%		\addlegendentry{\textbf{BW}}
		
		\addplot+table[x=States, y=SBW, col sep=tab]
		{content/Experiments/graphdata/set35.csv};
%		\addlegendentry{\textbf{Sparse BW}}
		
		\addplot+table[x=States, y=GE, col sep=tab]
		{content/Experiments/graphdata/set35.csv};
%		\addlegendentry{\textbf{Greedy Extend}}
	\end{axis}
\end{tikzpicture}
\label{fig:dataset35}
\end{minipage}
\\
\begin{minipage}[t]{\dimexpr0.6\textwidth-2\tabcolsep}
\end{minipage}
&
\begin{minipage}[t]{\dimexpr0.4\textwidth-2\tabcolsep}
\end{minipage}
\end{tabular*}%
	
Initial observations
\begin{itemize}
\item BW performs the best on 3 out of 4 sets
\item 
\item 
\end{itemize}	
	
%\subsection{Greedy Extend Experiments}
	%\subsection{Greedy Extend}

The first big question about the Greedy Extend algorithm, is how the choice of $\beta$ affects the performance of the algorithm.
As $\beta$ denotes the iterations of Baum Welch to run each time the algorithm attempts to extend the graph, increasing $\beta$ will also increase the run time of the algorithm. It may be the case that better results are achieved when $\beta$ is increased, since more iterations of Baum Welch also means a greater increase in likelihood. However, it could be the case that using many iterations early increases the chance of getting trapped in a local optimum.
An experiment has been conducted of using different values for $\beta$ on data set $1$ from the Pautomac competition. The results can be seen in figure \ref{fig:ge-different-thresholds-tested}. Each line represents an average over 5 runs of Greedy Extend with the specified number of iterations.
  
\begin{tikzpicture}
\begin{axis}[xlabel={$x$},ylabel={Column Data}]

\addplot table[x index=0,y index=1,col sep=tab] {content/Experiments/graphdata/ge-intermediate-iterations-test.csv};
\addlegendentry{Iterations: 0}
  
\addplot table[x index=0,y index=2,col sep=tab] {content/Experiments/graphdata/ge-intermediate-iterations-test.csv};
\addlegendentry{Iterations: 1}

\end{axis}
\end{tikzpicture}  

